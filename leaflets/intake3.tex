\documentclass{article}

% Language setting
% Replace `english' with e.g. `spanish' to change the document language
\usepackage[english]{babel}

% Set page size and margins
% Replace `letterpaper' with `a4paper' for UK/EU standard size
\usepackage[letterpaper,top=2cm,bottom=2cm,left=3cm,right=3cm,marginparwidth=1.75cm]{geometry}
% Useful packages
\usepackage{amsmath}
\usepackage{graphicx}
\usepackage[colorlinks=true, allcolors=blue]{hyperref}
\usepackage{braket}

% Custom commands
\newcommand{\refsec}[1]{Section~\ref{#1}}
\newcommand{\mat}[1]{\mathbf{#1}}


\title{intake:Structural analysis of intrinsically disordered proteins by small-angle X-ray scatteringw }
\subtitle{Pau Bernado ́ *a and Dmitri I. Svergun*b Received 4th July 2011, Accepted 2nd September 2011 DOI: 10.1039/c1mb05275f}
\begin{document}
i was told this paper elaborated on ensemble representation of IDPs
here is what the paper claims
\begin{items}
    \item small angle x ray scattering (saxx) is good at quantitatively analyzing IDPs 
    \item presents examples on combined approaches
    \item they can measure more than one conformation and get flexibility data and other structural info
    \item the coexistence of confoormations only works if you use the Ensemble Optimization Method (EOM)
\end{items}
\end{document}
