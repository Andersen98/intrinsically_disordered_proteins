\documentclass{article}

% Language setting
% Replace `english' with e.g. `spanish' to change the document language
\usepackage[english]{babel}

% Set page size and margins
% Replace `letterpaper' with `a4paper' for UK/EU standard size
\usepackage[letterpaper,top=2cm,bottom=2cm,left=3cm,right=3cm,marginparwidth=1.75cm]{geometry}
% Useful packages
\usepackage{amsmath}
\usepackage{graphicx}
\usepackage[colorlinks=true, allcolors=blue]{hyperref}
\usepackage{braket}

% Custom commands
\newcommand{\refsec}[1]{Section~\ref{#1}}
\newcommand{\mat}[1]{\mathbf{#1}}


\title{Towards a robust description of intrinsic protein disorder using nuclear magnetic resonance spectroscopyw}
\subtitle{Robert Schneider, Jie-rong Huang, Mingxi Yao, Guillaume Communie, Vale ́ ry Ozenne, Luca Mollica, Loı ̈ c Salmon, Malene Ringkjøbing Jensen and Martin Blackledge* Received 14th July 2011, Accepted 8th August 2011 DOI: 10.1039/c1mb05291h}
\begin{document}
i was told this paper elaborated on ensemble representation of IDPs
here is what the paper claims
\begin{items}
    \item given advancements in the study of IDPs, they hope to construct a unified molecular description of the disordered state.this unified description uses different datasets to describe the conformal behavour of the disordered state
    \item this is a review of recent tools developed to describe idps and select representative ensembles.
    \item each nmr parameter is sensative to different structural and dynamical behaviors of the disordered state
    \item each experimental nmr parameter is sensitive to different aspects of the IDPs structural and dynamical behaviour
    \item each nmr parameter will have its own relevant averaging properties off the specific physical interaction taking place
\end{items}
\section{Key Equations}

\begin{equation}\label{eq:residual-dipolar-couplings}
    D_{ij} = - \fraq{\gamma_i \gamma_j \hbar \mu_0}{8\pi^2r^3}\left \langle \frac{3 cos^2 \Omgega -1}{2} \right \rangle
\end{document}
