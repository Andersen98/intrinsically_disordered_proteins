\documentclass{article}

% Language setting
% Replace `english' with e.g. `spanish' to change the document language
\usepackage[english]{babel}

% Set page size and margins
% Replace `letterpaper' with `a4paper' for UK/EU standard size
\usepackage[letterpaper,top=2cm,bottom=2cm,left=3cm,right=3cm,marginparwidth=1.75cm]{geometry}
% Useful packages
\usepackage{amsmath}
\usepackage{graphicx}
\usepackage[colorlinks=true, allcolors=blue]{hyperref}
\usepackage{braket}

% Custom commands
\newcommand{\refsec}[1]{Section~\ref{#1}}
\newcommand{\mat}[1]{\mathbf{#1}}


\title{Intrinsically Disordered Proteins}
\author{Maeve Andersen}

\begin{document}
\maketitle

\begin{abstract}
    The structure and dynamics of intrinsically disordered proteins (IDPs) represent an active field of research in biophysics.
    Previous methods developed to describe protein structure and dynamics assume the protein to be ordered, in that the protein has a native conformation with some conformational varDescribing IDPs quantitatively requires many more degrees of freedom than experiment alone can provide.
    Thus, novel approaches to quantifying IDPs have been developed that take advantage of both computational theory and experiment. 
 In this paper, I will review IDPs. In particular I will review their core phenomena, their dynamics via a toy model of protein transitions using the energy landscape model, how IDPs are measured using Nuclear Magnetic Resonance, and finally how IDPs are studied today using a combination of computational methods and experimental measurements.
\end{abstract}


\section{Introduction}

Intrinsically disordered proteins (IDPs) represent an active area of research in protein science.
Characterization of IDPs is important as they are involved in cellular signaling and regulation,\cite{wrightIntrinsicallyDisorderedProteins2015}
and are associated with human diseases, such as neurodegenerative disease, cardiovascular disease, amyloidoses, cancer, and diabetes.\cite{uverskyIntrinsicallyDisorderedProteins2008}
Although challenging, modern methods of characterizing IDPs can provided new insights to crutial protein function human biological mechanisms. \cite{bonomiSimultaneousDeterminationProtein2018}.

\section{Conformational Ensembles}

It is important to note that the term "ensemble" is often not carefully used within the field of structural biology.
In statistical mechanics, a protein's ensemble, given some set of conditions, is a probabability distribution over all possible conformations of that protein.
In structural biology, the term is overloaded with several meanings.\cite{gaalswykEmergingRolePhysical}
There is of course a thermodynamic ensemble, which refers to the distribution of a protein's conformations under thermal equilibrium.
Additionally there are uncertainty ensembles, which are collections of conformations that are degenerate due to sparse, ambiguous, or noisy data.
Uncertainty ensembles may be further broken down depending on their source.
For example, they may come from an arbitrary algorithm, or they may come from a well defined distribution.\cite{gaalswykEmergingRolePhysical}

This section will review how IDPs are represented by "ensembles". 

\bibliographystyle{alpha}
\bibliography{cryo-em,ensembles,nmr,motivation}

\end{document}
